\documentclass{article}
\usepackage{hyperref}
\title{Project Proposal}
\author{Elijah Moore}

\begin{document}

\maketitle
\section{Status update notes}
    I am not sure where to begin building a model for this particular task. I am rethinking the reinforcement learning method, as we have not discussed that yet.
    Thus I am thinking I may instead go with an RNN To build the model. Additionally,
    I have little time left to devote to this project after doing the homework and work for my other classes.

    With that being said, I am doing my best with the power of the internet!

    https://www.datacamp.com/tutorial/lstm-python-stock-market

\section{Introduction}

   Through the use of stock trading algorithms, large firms are able to take advantage of market corrections
   to turn a profit. The stock market/valuation of financial assets could be considered something of a stochastic system.
   
   My project uses some of the vast amounts of data available on finances to
   create buy/sell indicators for stock trading.

\section{Dataset Proposal}
    I use yfinance's ticker module to access trading data on stocks.
    This yfinance submodule allows access to a larger number of input parameters than merely the price of the stock at a given moment.
    It also includes data about trades that are being made by `insiders' of a company, which will be useful for future iterations of this project.

    The following stocks are evaluated based off of daily closing price over the last two years.
    \begin{enumerate}
        \item Apple (AAPL)
        \item Tesla (TSLA)
        \item Microsoft (MSFT)
        \item Nvidia (NVDA)
        \item Ford (F)
        \item Google (GOOGL)
    \end{enumerate}
\section{Design of experiment and evaluation of results}

    The intent is to use a reinforcement learning model to devise an optimal strategy to trade stocks.
    The method for this will be to reserve Apple for final evaluation, and use the rest for training the model.
    
    The evaluation will be based off of the Quantity of money made by the strategy chosen.

    % I have a copy of the entire Bee Movie, which has been compressed, and reduced down to 49.5 Megabytes.
    % The normal size of this movie is much higher. 

    % I would like to use Matlab to improve the quality of this recording, and increase the resolution back to 1980 * 1080 Pixels.
    % Additionally, I am sure the audio could use some touching up and resampling to a higher sample rate.
    
    % The Idea which I would like to implement is to increase the quality of a crappy recording of a movie.

    % This will involve
    
\section{Appendix}
    The code is available
    \hyperlink{https://github.com/Elijawhan/CPE490PROJECT}{Here}
\subsection*{References: }
    \hyperlink{https://www.datacamp.com/tutorial/lstm-python-stock-market}{www.datacamp.com}
\end{document}