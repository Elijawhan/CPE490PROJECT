\documentclass{article}

% ready for submission
\usepackage[final]{neurips_2023}

\usepackage[utf8]{inputenc} % allow utf-8 input
\usepackage[T1]{fontenc}    % use 8-bit T1 fonts
\usepackage{hyperref}       % hyperlinks
\usepackage{url}            % simple URL typesetting
\usepackage{booktabs}       % professional-quality tables
\usepackage{amsfonts}       % blackboard math symbols
\usepackage{nicefrac}       % compact symbols for 1/2, etc.
\usepackage{microtype}      % microtypography
\usepackage{xcolor}         % colors


\title{Formatting Instructions For NeurIPS 2023}

\author{%
  Elijah Moore \\
  Department of Electrical and Computer Engineering\\
  University of Alabama in Huntsville\\
  Huntsville, AL 35805 \\
  \texttt{elijah.moore@uah.edu} \\
}


\begin{document}


\maketitle


\begin{abstract}
  
\end{abstract}

\section{Introduction}

    Through the use of stock trading algorithms, large firms are able to take advantage of market corrections
    to turn a profit. The stock market/valuation of financial assets could be considered something of a stochastic system.
   
    This project uses some of the vast amounts of data available on finances to
    predict future trends in the stock market.

\section{Design of Experiment}
    Yfinance's ticker module is used to access trading data on stocks.
    This yfinance submodule allows access to a larger number of input parameters than merely the price of the stock at a given moment.
    It also includes data about trades that are being made by `insiders' of a company, which would be useful for future iterations of this project.

    The Apple (AAPL) stock is evaluated based off of daily closing price over the last three years.

    The intent is to use a Long Short Term Model (LSTM) to predict the next days stock value.
    

% \begin{figure}
%   \centering
%   \fbox{\rule[-.5cm]{0cm}{4cm} \rule[-.5cm]{4cm}{0cm}}
%   \caption{Sample figure caption.}
% \end{figure}




\begin{ack}

    No outside funding was recieved for this.


\end{ack}


\section*{References}


{
\small

}

%%%%%%%%%%%%%%%%%%%%%%%%%%%%%%%%%%%%%%%%%%%%%%%%%%%%%%%%%%%%


\end{document}